\documentclass[ngerman, parskip=half, titlepage=false]{scrartcl}

%% encodings                                        
\usepackage[T1]{fontenc}                            
\usepackage[utf8]{inputenc}                         
\usepackage[shorthands=off]{babel}                  
\usepackage{lmodern}                                
\usepackage{microtype}                              
\usepackage{scrlayer-scrpage}                       
\usepackage{enumerate}                              
\usepackage{csquotes}                               
%% maths                                            
\usepackage{amsmath}                                
\usepackage{mathtools}                              
\usepackage{amsthm}
\usepackage{amssymb}                                 
\usepackage{dsfont}
\usepackage{mathrsfs}

\usepackage{wasysym}

%commutative diagramms
\usepackage{tikz-cd}
%\usetikzlibrary{babel}

%links                                              
\usepackage{hyperref}                               
                                                    
%special mathsymbols                                
\newcommand{\C}{\mathds{C}}                         
\newcommand{\R}{\mathds{R}}                         
\newcommand{\N}{\mathds{N}}                         
\newcommand{\Z}{\mathds{Z}}                         
\newcommand{\Q}{\mathds{Q}}
%\newcommand{\P}{\mathcal{P}}
\newcommand{\p}{\mathfrak{p}}

                                                    
%%%%%%%%%%%%%%%%%%%%                                
%theoremstyle                                       
%\theoremstyle{definition}                           
%\newtheorem*{Def}{Defintion}                        
%\theoremstyle{plain}                               
%\newtheorem{Beh}{Beh}                              
%\newtheorem{Vor}{Vor}                                                    
                          
% math (theorems)                     
\theoremstyle{definition}             
\newtheorem{Def}{Definition}[section] 
\newtheorem{Bsp}[Def]{Beispiel}       
\theoremstyle{remark}               
\newtheorem{Bem}[Def]{Bemerkung}      
\newtheorem{Wdh}[Def]{Wiederholung}   
\theoremstyle{plain}                  
\newtheorem{Satz}[Def]{Satz}          
\newtheorem*{satz}{Satz}              
\newtheorem{Lem}[Def]{Lemma}          
\newtheorem{Kor}[Def]{Korollar}       
\newtheorem{Fol}[Def]{Folgerung}      
\newtheorem{Beh}[Def]{Behauptung}
\newtheorem{Auf}[Def]{Aufgabe}
                          
%stuff used only a few times                        
\DeclareMathOperator{\f}{\varphi}                         
\newcommand{\spec}[1]{Spec \left( #1 \right)}
% \newcommand{\exxp}[1]{\exp \left( #1 \right)}       
% \renewcommand{\sin}[1]{\sin \left( #1 \right)}      

%own color for proof environment
\definecolor{mygray}{rgb}{0.2,0.2,0.2}

 \makeatletter
    \renewenvironment{proof}[1][\proofname]{\par
    	\pushQED{\hfill$\blacksquare$}%
    	\slshape\color{mygray}\topsep6\p@\@plus6\p@\relax
    	\noindent%
    	\raisebox{\baselineskip}{\rlap{\bfseries%
    			#1\@addpunct{.}}}\ignorespaces%TEst
    }{%
    \popQED\@endpefalse
}
\makeatother

%title                                            
% \title{Übungsblatt ??}                            
% \subtitle{Theo, Gruppe 1}                         
% \author{Julian Seipel}                            
% \pagestyle{scrheadings}                           



%%% Local Variables:
%%% mode: latex
%%% TeX-master: "main"
%%% End:


\begin{document}

\part{Aufgabensammlung}

\section{Ringe}

\begin{Satz}
  Sei $R$ ein endlicher Ring, so sind äquivalent:\\
  \begin{itemize}
  \item[i)] $R$ ist Körper
  \item[ii)] $R$ ist Integritätsring
  \end{itemize}
Hinweis: Prüfe das jedes Primideal schon maximal ist.
\begin{proof}
  \begin{description}
    \item[$i) \Rightarrow ii)$] klar
    \item[$ii) \Rightarrow i)$] Da $R$ ein endlicher Ring ist, so
      betrachte für ein beliebiges Element $a \in R \setminus \{ 0 \}$
      die Menge $\{ a^n | n \in \N \} \subset R$. Diese Menge ist
      endlich als Teilmenge einer endlichen Menge, somit gibt es
      $n,m \in \N$ mit $n \neq m$ so dass gilt $a^n = a^m$. OBdA gilt
      $n > m$ , so folgt $a^m (a^{n-m} - 1) = 0$. Da $R$ ein Integritätsring ist
      gilt $a^m = 0$ oder $a^{n-m} = 1$. Bei letzteren Fall ist man fertig, bei 
      erstem falls ist noch zu zeigen, das jedes Element in einem endlichen Ring
      entweder Einheit oder Nullteiler. Also sind alle Nullteiler schon
      nilpotent. Aber da $R$ integer ist, folgt die Behauptung.
  \end{description}
\end{proof}
\end{Satz}

\begin{Satz}
  Sei $R \neq 0$ Ring und $\mathcal{M} \subset R$ ein maximales Ideal.
  So ist $R/\mathcal{M}^n$ lokaler Ring, für alle $n \in \N_{\geq 1}$
\end{Satz}

\begin{Satz}
  Zeige: Jeder Unterring $S \subset \Q$ ist eine Lokalisierung von $\Z$
\end{Satz}

\begin{Def}
  Sei $R$ Ring, so definierte das \textbf{Radikalideal} von $R$, als
  \begin{gather*}
    rad(R) \coloneqq \left\{ r \in R \; \Big| \; \exists n \in
      \N_{\geq 1} : r^n = 0 \right\}
  \end{gather*}
  Dieser Ring heißt \textbf{reduziert}, falls gilt:
  \begin{gather*}
    rad(R)=\sqrt{R} =0
  \end{gather*}
\end{Def}

\begin{Satz}[Universelle Eigenschaft des Radikalideals]
  Sei $R$ Ring, und $R_{red} = R/rad(R)$. Zeige 
  \begin{itemize}
    \item[1)] $R_{red}$ ist reduziert
      
    \item[2)] Jeder Ringhomomorphismus $\varphi : R \rightarrow
      R^{'}$, nach einem reduzierten Ring $R^{'}$, faktorisiert
      über die kanonische Projektion $\pi : R \rightarrow R_{red}$, 
      so dass gilt: $\exists ! \bar{\varphi} : R_{red} \rightarrow R^{'}$
      mit $\pi \bar{\varphi} = \varphi$
  \end{itemize}
\end{Satz}

\begin{Lem}
  Es gibt nur endlich viele Gruppen $G$ der Ordnung $n \in \N$
\end{Lem}

\begin{Satz}
  Sei $S \subset R$ beliebige Teilmenge eines Ringes und $I=<S>$, das
  Ideal von $S$ erzeugt, so gilt:\\
  \begin{gather*}
    I=R \Leftrightarrow \forall m : \text{maximales Ideal} : m \cap S 
    \neq \emptyset
  \end{gather*}
 
\end{Satz}

\begin{Satz}[Lokalisierung mit einem Element]                   
  Sei $R$ Ring, und $f \in R$ mit $f$ nicht nilpotent, so gilt  
  die Isomorphie (wobei $R_f$ die Lokalisierung nach $\left\{   
    f^n \; \Big| \; n \in N \right\}$ bedeutet)                 
  \begin{gather*}                                               
    R_f \simeq R[X]/(fX - 1)                                    
  \end{gather*}                                                 
\end{Satz}                                                      

\begin{Bsp}
  \begin{itemize}
    \item $\Z[\frac{1}{2}] \simeq \Z[X]/(2X-1)$
  \end{itemize}
\end{Bsp}

\begin{Lem}
  Sei $R$ Ring, $f \in R$ ein beliebiges Element und $n \in \N$ Zeige:\\
  \begin{gather*}
    \displaystyle R_f \simeq R_{f^n}
  \end{gather*}
  \begin{proof}
    Setze $S \coloneqq \{ f^k | k \in \N \}$ und $\hat{S} \coloneqq \{ f^{kn} 
    | k \in \N\}$. Es ist zu zeigen, dass die beiden multiplikativen Systeme
    die selbe Lokalisierung erzeugen.
  \end{proof}
\end{Lem}

\section{Topologische Räume}

\begin{Satz}
  Sei $X$ ein endlicher, hausdorffscher, topologischer Raum, so ist
  dieser diskret.
Anmerkungen: endlich $\rightarrow X$ als Menge endlich\\
diskret $\rightarrow$ die Topologie ist die Potenzmenge
\end{Satz}

\section{Ringe und anderes}

\begin{Lem}
  Sei $Q[\cos x , \sin x]$ der kleinste Unterring von $\R^\R$ der
  die beiden transzendenten Funktionen $\sin , \cos$ enthält. Zeige
  dass das allgemeine Element in diesem Unterring von der folgenden
  Form ist:
  \begin{gather*}
    a_0 + \sum\limits_{m=1}^n a_m \cos mx + b_m \sin mx
  \end{gather*}
\end{Lem}

\begin{Lem}
  Sei $n \in \N$.\\
  \begin{itemize}
    \item[1)] Jedes Ideal von $\Z/(n)$ ist Hauptideal
    \item[2)] Bestimme alle Radikalideale von $\Z/(n)$
  \end{itemize}
  \begin{proof}
    \begin{description}
      \item[1)] klar, aufgrund der additiven Struktur der Menge $\Z$
      \item[2)] Es ist zu zeigen das gilt:\\
        \begin{gather*}
          \left\{ p \subset \Z \Big| p \text{ Ideal } , \sqrt{p} = p
          \right\} =  \left\{ \bigcap_i (p_i) \Big| n \in \N ,
            \forall i \leq n : p_i \in \spec{R} \right\}
        \end{gather*}
        Da $\Z$ ein noetherscher Ring ist, gibt es eine eindeutige Zerlegung
        jedes Ideal in primäre Ideale. Da $\Z$ Hauptidealring ist, folgt
        unmittelbar das die primären Ideale genau die Potenzen von den 
        Primidealen sind. Also hat ein Ideal $I \subset \Z$ die Form.
        \begin{gather*}
          I = (n) = \; \bigcap\limits_{p | n} \; (p^{n_p} ) \;
        \end{gather*}
        Da nun die Radikalbildung mit dem endlichen Schnitt vertauscht und
        $\sqrt{(p^n)} = (p) $ für eine Primzahl $p \in \N$. Somit folgt die
        Behauptung.
      \end{description}
  \end{proof}
\end{Lem}

\begin{Satz}
  Sei $R$ ein (allgemeiner) Ring, hier im Allgemeinen nicht kommutativ
  und/oder mit Eins, aber immer assoziativ, zeige oder wiederlege
  durch Gegenbeispiele folgende Behauptungen:
  \begin{itemize}
    \item[1)] Es existiert ein Ring $S$ mit Eins und ein Ringhomomorphismus 
      $\varphi : R \rightarrow S$
    \item[2)] Es existiert ein Ring $S$ mit Eins und ein Monomorophismus
      $\psi : R \rightarrow S$
  \end{itemize}
\end{Satz}

\begin{Satz}
  Sei $n \in \N$, so sind folgende Aussagen äquivalent:\\
  \begin{itemize}
    \item[i)] $Z/(n)$ ist lokaler Ring
    \item[ii)] $n$ ist Primpotenz, also gilt $\exists m,p \in \N : $ 
      $p$ ist prim: $n=p^m$
    \end{itemize}
\end{Satz}

$R$ ist Ring mit eins, und $I \subset R$ ein Ideal, definiere $D(R,I) \coloneqq
\left\{ (a,b) \in R \times R \Big| a-b \in I  \right\}$. Zeige $D(R,I)$ ist
Unterring von $R$.

$ 0 \in D(R,I)$ ist klar.\\

$(a_1,b_1),(a_2,b_2) \in D(R,I)$, so folgt $(a_1,b_1)*(a_2,b_2) = (a_1a_2,b_1b_2)$
so betrachte $a_1a_2 - b_1b_2$, mit $\exists i_1,i_2 \in I: a_1 = b_1 + i_1$ und
$a_2 = b_2 + i_2$, so folgt\\

\begin{Satz}
  Sei $F$ die frei Gruppe mit den Erzeugern $a,u$, betrachte den
  Quotienten:
  \begin{gather*}
    Q \coloneqq F/\{ ua=e, au=u^2a^2 \}
  \end{gather*}
  Zeige, das gilt: $uaa \neq auu$
\end{Satz}

\begin{Lem}
  Sei $R$ lokaler Ring, und $n \in \N$.\\
  So gilt für $A \in GL(n,R)$, in jeder Zeile oder Spalte befindet
  sich mindestens eine Einheit.
\end{Lem}

\begin{Lem}
  Seien $n, m \in \Z$ , so gilt für den Idealquotienten, für zwei
  Ideale $I,J \subset \R$ von einem Ring $R$
  \begin{gather*}
    \left( I:J \right) = \left\{ r \; \Big| \; r J \subset I   \right\}
  \end{gather*}
  Es gilt nun in $\Z$
  \begin{gather*}
    (n:m) = (m / gcd(n,m))
  \end{gather*}
\end{Lem}

\begin{Beh}
  Sei $R$ Ring und $p,q \in \spec{R}$, Primideale. So soll gelten:
  \begin{gather*}
    p,q \text{ minimale Primideale} \Rightarrow p,q \text{ coprim}
  \end{gather*}    
\end{Beh}

\begin{Lem}
  Seien $n,m \in \Z$ so gilt: $\Z/(n) \otimes_{\Z} \Z/(m) \simeq \Z/(gcd(n,m))$
\end{Lem}

\begin{Lem}
  Seien $N,M$ $R$-Moduln, und $N$ ein freies Moduln, also
  $N \simeq R^{(I)}$ für eine Indexmenge $I$, so gilt:
  \begin{gather*}
    N \otimes_R M \simeq M^{(I)}
  \end{gather*}
  \begin{proof}
    Es gilt, aufgrund dessen das $N$ ein freier Moduln über die
    Indexmenge $I$ ist.  Es gibt nun ein freies erzeugenden System
    $(x_i)_{i \in I}$, so folgt aufgrund der Vertauschung von dem
    Tensorprodukt und der direkten Summe, so gilt mit $Rx \simeq R$
    für ein $x \in M$, als $R$- Moduln
    \begin{gather*}
      N \otimes_R M 
      \simeq
      ( \bigoplus_{i \in I} R x_i ) \otimes_R M
      \simeq
      \bigoplus_{i \in I} ( R x_i \otimes_R M)
      \simeq
      \bigoplus_{i \in I} ( R  \otimes_R M)
      \simeq
      \bigoplus_{i \in I} M
      \simeq
      M^{(I)}
    \end{gather*}
  \end{proof}
\end{Lem}

\begin{Beh}
  Sei $A$ eine abelsche Gruppe, also $\Z$-Modul, so soll gelten:\\
  $A \otimes_\Z \Q = 0 \Leftrightarrow$ $A$ ist Torsionsgruppe.\\
  Für $A$ ist Torsionsgruppe, falls gilt: $\forall a \in A \; \exists n \in \Z \;
  na = 0$
\end{Beh}

\begin{Lem}
  Sei $(A_i)_{i \in I}$ eine Familie von Torsionsgruppen, so gilt:
  $A = \bigoplus_{i \in I} A_i$ ist Torsionsgruppe.
  \begin{proof}
    Sei $(a_i)_{i \in I} \in A$, so gilt für jeden Eintrag dieser
    Familie von Elementen aus den Torsionsgruppen. Für jedes $i \in I$ existiert
    $n_i \in \Z$ so dass $n_i a_i = 0 $. Da in direkten Summe nur endliche viele 
    Elemente von Null verschieden sind, setze
    \begin{gather*}
      n \coloneqq \prod\limits_{i \in I} n_i
    \end{gather*}
    dieses Produkt ist wohldefiniert, bzw endlich, nach vorherigen Argument der
    direkten Summe.
    so gilt nun $n (a_i)_{i \in I} = (n a_i)_{i \in I} = ( 0 )_{i \in I} $
  \end{proof}
\end{Lem}

\begin{Satz}
  Zeige folgende Aussagen:
  \begin{enumerate}[a)]
  \item  Sei $(R,m)$ ein lokaler, noetherscher Ring, zeige: \\
    Für $p \subset R$ $m$-primäres Ideal so ist $R/p$ ist artinsch.
  \item Die direkte Summe von artinschen Ringen ist wieder artinsch
  \end{enumerate}
  \begin{proof}
    \begin{description}
    \item[a)] Es gilt zunächst, das $m = \sqrt{p}$ gilt, da $R/p$ als Quotient eines
      noetherschen Ringes wieder noethersch ist, ist das maximal Ideal endlich  
      erzeugt. Also existiert ein $n \in \N$ mit $\pi(m^n) = 0$
      %TODO, warum ist das bild des maximalen ideals im faktorring R/p maximal
      %TODO warum ist der faktorring lokal?
    \end{description}
  \end{proof}
\end{Satz}

\begin{Satz}
  Seien $R_1,R_2,\ldots,R_n$ Ringe, so gilt:
  \begin{gather*}
    \spec{\prod\limits_{i = 1}^n R_i} \simeq \coprod\limits_{i=1}^n \spec{R_i}
  \end{gather*}
  \begin{proof}
    Es gilt für die Primideale aus $R_1 \times \ldots \times R_n$ das sie von 
    der folgenden Gestalt sind:
    \begin{gather*}
      R_1 \times \ldots \times p_i \times \ldots \times  R_n
    \end{gather*}
    für ein $p_i \in \spec{R_i}$. Die Gestalt des Koprodukts ist wie folgt:
    \begin{gather*}
      \coprod\limits_{i} R_i \; = \;  \prod\limits_i (i,R_i)
    \end{gather*}
    So folgt die Behauptung durch folgende kanonische Abbildung:\\
    \begin{center}
    \begin{tikzcd}
      \spec{\prod\limits_{i = 1}^n R_i} 
      \ar[r]
      & \coprod\limits_{i=1}^n \spec{R_i}
    \end{tikzcd}\\
    \begin{tikzcd}
      R_1 \times \ldots \times p_i \times \ldots \times R_n \ar[r,
      mapsto] & (i,p_i)
    \end{tikzcd}
  \end{center}
\end{proof}
\end{Satz}

\begin{Beh}
  Sei $R$ ein Ring und $Q(R)$ der totale Quotientenring des Ringes $R$
  \footnote{Dies ist die Lokalisierung nach allen Nichtnullteiler},
  so gibt es zu  jedem Unterring von $T \subset Q(R)$, ein multiplikatives
  System $S \subset R$, so dass gilt $T = R_S$.
  Also ist jeder Unterring des totalen Quotientenrings eine Lokalisierung des 
  Ringes $R$. (Betrachte zunächst $R = \Z$)
\end{Beh}

\begin{Satz}[Allgemeines Nakayama Lemma]
  Sei $R$ Ring, $I \subset R$ Ideal und $M$ ein $R$-Modul so gilt:\\
  \begin{gather*}
    M = IM \Rightarrow  \exists a \in I \; : \; (1+a)M=0
  \end{gather*}
\end{Satz}

\begin{Lem}
  Sei $M$ ein endlich erzeugter $R$-Modul. So gilt:
  Jeder surjektiver Endomorphismus ist schon Automorphismus.
\end{Lem}

\begin{Satz}
  Sei $M$ ein $R$-Modul, so sind folgende Aussagen äquivalent:\\
  \begin{enumerate}[i)]
    \item Es existieren $n,m \in \N$ so dass eine exakte Sequenz existiert:\\
      \begin{tikzcd}
        R^n \ar[r] & R^m \ar[r] & M \ar[r] & 0
      \end{tikzcd}
    \item Es existiert $\pi : R^n \rightarrow M$, surjektiv und
      $\ker \pi$ endlich erzeugt.
    \end{enumerate}
  \end{Satz}

\begin{Beh}
  Sei $R$ ein nicht noetherscher Ring, so existiert ein $R$-Modul $M$ der
   noethersch ist. Dieser muss nicht endlich erzeugt sein.
\end{Beh}

\begin{Satz}
  Es gilt für einen endlichen kommutativen Ring $R$ mit Eins:\\
  \begin{enumerate}[a)]
    \item $R$ ist endlicher also auch endlich erzeugter $\Z$-Modul
    \item $R$ ist als $\Z$-Modul artinsch und noetherscher Ring
    \item $R$ ist endliches Produkt von lokalen, artinschen Ringen
  \end{enumerate}
\end{Satz}

\begin{Beh}
  Sei $R$ Ring und $m \subset R$ ein Ideal, so sind folgende Aussagen äquivalent:\\
  \begin{enumerate}[i)]
    \item $m$ ist maximales Ideal
    \item $\forall f \in R  \setminus m : (f) + m = (1) $ 
  \end{enumerate}
  \begin{proof}
    \begin{description}
      \item[$i) \Rightarrow ii)$] Da $m$ maximal ist, gilt für ein Element $f$ das 
        nicht aus $m$ ist, das $(f) + m$ ein Ideal ist, dass das maximale Ideal
        $m$ enthält und nach der Maximalität nur das $(1)$ Ideal sein kann.
      \item[$ii) \Rightarrow i)$] Zeige diese Richtung indirekt, hierzu ist
        nur mindestens ein Element $f \in R \setminus m$ anzugeben, so das für
        nicht maximale Ideal $m$ die Summe $(f) + m $ ungleich dem ganzen Ring
        ist. Dies folgt leicht daraus, da $(f) + m$ als Ideal in einem maximalem
        Ideal enthalten ist, und somit nicht der ganze Ring ist.
    \end{description}
  \end{proof}
\end{Beh}

\begin{Beh}
  Seien $R_1,R_2$ kommutative, noethersche Ringe mit endlicher
  Krulldimension. So \textbf{sollen} folgende Aussagen
  gelten:\\
  \begin{enumerate}[a)]
    \item $dim(R_1 \times R_2) =  \max \{ dim(R_1), dim(R_2) \} $
  \end{enumerate}
\end{Beh}

\begin{Lem}
  In einem artinischen, noetherschen  Ring ist das Nilradikal nilpotent.
\end{Lem}

\begin{Lem}
  Sei $I$ eine partiell geordnete Menge, so sind folgende Aussagen äquivalent:\\
  \begin{enumerate}[i)]
    \item Jede echt aufsteigende Kette wird stationär
    \item Jede nichtleere Teilmenge $J \subset I$ hat ein maximales Element
  \end{enumerate}
  Es gilt die Aussage auch, falls man aufsteigend durch absteigend und maximales
  durch minimales ersetzt.
\end{Lem}

\begin{Satz}
  Seien $M,N$ endlich erzeugte  $R$-Moduln, über einen noetherschen Ring $R$.
  So ist der $R$-Modul $Hom_R(M,N)$ endlich erzeugt.
\end{Satz}

\begin{Kor}
  Sei $M$ ein $R$-Modul, endlich erzeugt über einem noetherschen Ring $R$.
  So ist der duale Modul $Hom_R(M,R)$ noethersch.
\end{Kor}

\begin{Lem}
  Sei $R$ ein artinscher Ring, so ist $\spec{R}$ diskret (mit der Zariski-Toplogie).
  \begin{proof}
    Da $R$ ein artinischer Ring ist, gilt das $\spec{R}$ endliche Menge ist, 
    und mit dem maximal Spektrum übereinstimmt. Nun ist jeder einzelne Punkt
    abgeschlossen, somit lässt sich jede daraus vereinigen, und es folgt die 
    Behauptung.
  \end{proof}
\end{Lem}

\begin{Lem}
  Sei $R$ ein Ring und $p \in \spec{R}$ minimal bzgl. Inklusion. So ist
  $p$ eine Teilmenge der Nullteiler des Ringes $R$.
\end{Lem}

\begin{Beh}
  Sei $R$ Ring und $\Sigma$ die Menge aller multiplikativen Systeme
  von $R$, die nicht die $0$ enthalten. So ist $\Sigma$ durch die
  Inklusion partiell geordnet und nach dem Zornsches Lemma existieren
  maximale Elemente.
  Zeige nun für $S \in \Sigma$:\\
  $S$  maximal $\Leftrightarrow$ $R \setminus S $ ist minimales Primideal.\\
  Wobei die Minimalitäten stets bzgl der Inklusion zu verstehen sind.
  Sei $S \in \Sigma$, so ist $S$ \textbf{saturiert}, falls gilt:\\
  $ab \in S \Leftrightarrow a \in S \text{ und } a \in S$\\
  Zeige nun:\\
  $S$ saturiert $\Leftrightarrow$ $R \setminus S$ ist eine Vereinigung
  von Primidealen.\\
  Zeige das die Menge aller Nullteiler eine Vereinigung von Primidealen ist.
\end{Beh}

\begin{Lem}
  Sei $R$ Ring, so zeige folgende Aussage:\\
  Für alle $p \in \spec{R}$ ist $rad(R_p)=0$, dann gilt $rad(R) = 0$
\end{Lem}

\begin{Beh}
  Es existiert ein Ring $R$, so dass $R_p$ noethersch ist, für alle $p \in \spec{R}$
  aber $R$ nicht noethersch ist.
\end{Beh}

\begin{Satz}
  Sei $R$ ein noetherscher Ring, so sind folgende Aussagen äquivalent:\\
  \begin{enumerate}[i)]
    \item $R$ ist artinsch
    \item $\spec{R}$ ist diskret und endlich
    \item $\spec{R}$ ist diskret
  \end{enumerate}
  \begin{proof}
    \begin{description}
      \item[$i) \Rightarrow ii) \Rightarrow iii)$] klar
      \item[$iii) \Rightarrow i)$] Zeige das $dim(R)=0$ gilt oder nutze
        die Quasikompaktheit von dem Spektrum aus.
    \end{description}
  \end{proof}
\end{Satz}

\begin{Satz}
  Zeige, das falls in einem Ring $R$ jedes Primideal endlich erzeugt ist,
  das der Ring schon noethersch ist.\\
  Tipp: Betrachte die Menge $\Sigma$ aller nicht endlich erzeugten Ideale von $R$
  und zeige das die (existierenden) maximalen Elemente Primideale sind.
\end{Satz}

\begin{Lem}
  Sei $R$ Ring. Es gilt $\forall x \in R \; \exists n > 1 : x^n = x$ , dann
  ist jedes Primideal schon maximal.
\end{Lem}

\begin{Beh}
  Sei $R$ lokaler Ring, $M,N$ endlich erzeugte $R$-Moduln, so gilt:\\
  $M \otimes N = 0 \Rightarrow M = 0 \text{ oder } N = 0$
\end{Beh}

\begin{Lem}
  Es gilt in einem Ring $R$ mit den Idealen $I,J \subset R$ folgende Äquivalenz:
  \begin{enumerate}[i)]
    \item $V(I)=V(J)$
    \item $\sqrt{I} = \sqrt{J}$
  \end{enumerate}
\end{Lem}

\begin{Lem} 
  \begin{enumerate}[1)]
  \item Sei $G$ eine Gruppe und $U_\alpha$ die Familie aller
    Untergruppen von $G$ so gilt: $\varinjlim\limits_{\alpha} U_\alpha = G$.
    Hinweis: Zeige zuerst dass das System aller Untergruppen auf kanonische Art
    zu einem direkten System wird, durch Inklusion.
  \item Sei $M$ eine Menge und $(M_\alpha, \vartheta_{\alpha\beta})$
    ein direktes System von Teilmengen die durch die Inklusion zu
    einem direkten System wird, zeige:
    $\lim\limits_{\alpha} M_\alpha = \bigcup\limits_\alpha M_\alpha$
  \end{enumerate}
\end{Lem}

Eigenschaften die Ringe haben können:
\begin{enumerate}[a)]
  \item endlich
  \item noethersch
  \item kommutativ
  \item artinsch
  \item integer
  \item lokal
  \item (Körper)
  \item faktoriell
\end{enumerate}

Konstruktion von Ringen
\begin{enumerate}[a)]
  \item direkte Summe $(\bigoplus)$
  \item Lokalisierung nach einem multiplikativen System  $S^{-1}R = R_S$
  \item Quotientenbildung nach einem Ideal $R/I$
  \item Gruppenring, bzgl einer Gruppe $R[G]$
  \item Tensorprodukt von Ringen, wenn man diese als $R$-Modul über
    den selben Ring auffasst, zb. kann man jeden Ring als $\Z$-Modul auffasen
\end{enumerate}

\begin{Beh}
  Sei $R[X]$ ein Ring über einen noetherschen Ring $R$ dann ist nach
  dem Hilbertbasissatz jedes Ideal aus dem Ring endlich erzeugt. Finde
  zu folgenden Ideale endliche Erzeugendensysteme:\\
  \begin{enumerate}[1)]
  \item $R=\R$ und $I = \left( X + p \; | \;  p \in \N \text{ prim }  \right)$
  \item $R$
  \end{enumerate}
\end{Beh}

\begin{Beh}
  Sei $R$ Ring, so soll gelten, für $(a) \subset R$ Hauptideal ist jedes Ideal $J$
  das in $(a)$ enthalten ist, auch schon Hauptideal.
\end{Beh}

\begin{Lem}
  Sei $R$ ein lokaler, artinischer Ring, so hat dieser genau ein nilpotentes
  Primideal.
\end{Lem}

\begin{Beh}
  Für einen Integritätsring $R$, gilt für $p \in \spec{R}$ das es 
  eine kanonische Einbettung
  \begin{tikzcd}
    R_p \ar[r , hookrightarrow ] & Q(R)
  \end{tikzcd}
  in den Quotientenkörpern von $R$ gibt.
  \begin{proof}
    Verwende hierzu die universelle Eigenschaft der Lokalisierung nach
    $p$, denn jedes Element ausser der Null wird in den Quotientenkörpern 
    auf eine Einheit geschickt, somit muss diese Abbildung über die
    Abbildung 
    \begin{tikzcd}
      R \ar[r , "\tau" ] & R_p
    \end{tikzcd}
    faktorisieren.
  \end{proof}
\end{Beh}

\begin{Satz}
  Lokalisierung ist ein Funktor.
\end{Satz}

\begin{Beh}
  Sei $R=K[X_1,\ldots,X_n]$ der Polynomring in $n$ Variablen, so gibt es für
  ein Ideal $I$ mit den Erzeugern $f_1,\ldots,f_k \in R$, ein Verfahren
  um die Erzeugern von $\sqrt{I}$, zu bestimmen.
\end{Beh}

\begin{Lem}
  Seien $I_1,\ldots,I_n$ Ideale eines Ringes $R$, die paarweise koprim sind,
  so gilt: 
  \begin{gather*}
    \sum_i \prod_{j \neq i} I_j = R
  \end{gather*}
\end{Lem}

\begin{Satz}
  Sei $Z(p)$ die Zusammenhangskomponente eines top. Raums $X$ mit $p \in X$.
  Folgende Aussagen sind äquivalent:
  \begin{enumerate}[i)]
    \item $Z(p)$ offen
    \item $X$ ist lokal zusammenhängend für $\forall q \in Z(p)$
  \end{enumerate}
  Lokal zusammenhängend bedeutet, das zu jedem Punkt $p$ und jeder Umgebung $U$
  dieses Punktes, eine Umgebung $W$ von $p$ existiert, mit $p \in W \subset U$
  und $W$ zusammenhängend.
  \begin{proof}
    \begin{description}
      \item["$\Rightarrow$"] Sei $U$ eine Umgebung von $p$ $\Rightarrow$ $U'=
        Z(p) \cap U \subset U$ ist zusammenhängende Umgebung von $p$
      \item["$\Leftarrow$"] Sei $q \in Z(p)$ und $V$ eine Umgebung von $q$ 
        $\Rightarrow$ $\exists V'$ zusammenhängend so dass $q \in V' \subset V$
        Da $q \in Z(p) \exists U'$ zusammenhängend so dass $\{ p  \} \cup \{ q  \}
        \subset U'$ $\Rightarrow$  $V' \cup U'$ zusammenhängend und $p,q \in 
        U' \cup V'$ $\Rightarrow V' \subset V' \cup U' \subset Z(p) \Rightarrow
        Z(p)$   offen.
    \end{description}
  \end{proof}
\end{Satz}

\begin{Lem}[Zerlegung exakter Sequenzen in kurze exakte Sequenzen]
  Seien $A,A',A''$ Rechts $R$-Moduln über einen Ring, eine Sequenz
   $A_{\text{\Huge.}}$\\
   \begin{center}
     \begin{tikzcd}
       A' \ar[r, "f"] & A \ar[r, "g"] & A''
     \end{tikzcd}
   \end{center}
   und setze 
   \begin{itemize}
   \item $A_1 = ker(A' \rightarrow A) $
   \item $A_2 = im(A' \rightarrow A)$
   \item $A_3 = im(A \rightarrow A'')$
   \end{itemize}
   So sind folgende Aussagen äquivalent:\\
   \begin{enumerate}[i)]
     \item 
       \begin{tikzcd}
         A' \ar[r, "f"] & A \ar[r, "g"] & A''
       \end{tikzcd}
       ist exakt.
     \item Die folgenden kurzen Sequenzen sind exakt\\
       \begin{center}\begin{tikzcd}
         0 \ar[r] & A_1 \ar[r] & A' \ar[r] & A_2 \ar[r] & 0 \\
         0 \ar[r] & A_2 \ar[r] & A \ar[r] & A_3 \ar[r] & 0 \\
         0 \ar[r] & A_3 \ar[r] & A'' \ar[r] & A''/A_3 \ar[r] & 0
       \end{tikzcd}
     \end{center}
   \end{enumerate}
 \end{Lem}

\begin{Auf}
  Bestimme für $R'=\Z$ alle maximalen Ideale von $R=R'[X]$.
  Folgende Ideale aus $R'$ sind maximal:
  \begin{enumerate}
    \item $(X,p)$ für $p \in \Z$ prim 
  \end{enumerate}
\end{Auf}

\begin{Auf}
  Sei $\{ f_i : E \rightarrow E \}_{i \in I}$ Eine Familie von
  injektiven Abbildungen, gilt dann für die Verkettung
  $ f_1 \circ f_2 \circ \ldots$ dann auch das sie injektiv ist?
  Für $I$ endlich: ja.\\
  Für $I$ nicht endlich: nein.
  \begin{proof}
    Gegenbeispiel:\\ Setze $f(x) \coloneqq f_i (x) = ax+b$ für ein
    $0 < |a| < 1$, dann gilt:
    \begin{Beh}
      Sei $f_n = f \circ f \circ \ldots \circ f$ n-mal, dann gilt:\\
      $f_n(x) = a^n x + b \cdot \frac{a^{n+1} -1}{a-1}$
      \begin{proof}
        \begin{description}
        \item[$n=1$:] klar
        \item[$n \rightarrow n+1$:] Betrachte\\
          \begin{align*}
            f_{n+1}(x) = f(f_n(x)) 
          &= 
            f(a^n x + b \cdot \frac{a^{n+1}-1}{a-1}) 
          &= 
            a \cdot \left( a^n x + b \cdot \frac{a^{n+1}-1}{a-1} \right) + b        
          &= \\
            a^{n+1} x + b \cdot ( 1 + a \cdot \frac{a^{n+1}-1}{a-1})
          &= 
            a^{n+1} x + b \cdot (1 + a \cdot (1 + a + a^2 + \ldots a^{n}))
          &= \\
            a^{n+1} x + b \cdot (1  + a + a^2 + \ldots a^{n+1}))
          &= 
            a^{n+1} x + b \cdot \frac{a^{n+1}-1}{a-1}
          \end{align*}
        \end{description}
        \end{proof}
        Für die Bedingung $0 < |a| < 1$ ist $f$ injektiv, und für den
        Grenzwert gilt: $\lim\limits_{n \rightarrow \infty} f_n(x) = \frac{b}{1-a}$.
        Was konstant also nicht injektiv ist.
    \end{Beh}
  \end{proof}
\end{Auf}

\begin{Beh}
  Sei $A$ ein Integritätsring und $X = \spec{A}$ sein Spektrum. So gilt für jede
  offene, nichtleere Teilmenge von $X$, das sie dicht in $X$ liegt.
  \begin{proof}
    Da $A$ integer ist, gilt $(0) \in X$, betrachte nun $U \subset X$ offen, so 
    ist der Abschluss von $U$ gleich dem gesamten Raum.
  \end{proof}
\end{Beh}

\begin{Satz}
  Sei $R$ ein kommutativer Ring mit Eins, dann gibt es einen
  Homöomorphismus zwischen den Spektren $\spec{R}$ und
  $\spec{R_{red}}$, wobei $R_{red}=R/\sqrt{(0)}$.\\
  (Das Spektrum ignoriert die nilpotenten Elemente)
\end{Satz}


\end{document}

